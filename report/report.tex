\documentclass{article}
\usepackage[margin=1in]{geometry}

\usepackage[T1]{fontenc}

\usepackage{color}
\definecolor{lightgray}{rgb}{0.95, 0.95, 0.95}
\definecolor{darkgray}{rgb}{0.4, 0.4, 0.4}
\definecolor{purple}{rgb}{0.65, 0.12, 0.82}
\definecolor{editorGray}{rgb}{0.95, 0.95, 0.95}
\definecolor{editorOcher}{rgb}{1, 0.5, 0} % #FF7F00 -> rgb(239, 169, 0)
\definecolor{editorGreen}{rgb}{0, 0.5, 0} % #007C00 -> rgb(0, 124, 0)

\usepackage{titling}
\title{CS 5300 Project \#1}
\author{Jared Rainwater \& Samuel K. Grush}

\usepackage{fancyhdr}
\pagestyle{fancy}
\fancyhf{}
\lhead{\thetitle}
\rhead{\theauthor}
\cfoot{Page \thepage}

\usepackage{upquote}
\usepackage{listings}
\lstdefinelanguage{TypeScript}{
  morekeywords={break, case, catch, class, const, continue, debugger, default, delete, do, else, enum, export, extends, false, finally, for, function, if, import, in, instanceof, new, null, return, super, switch, this, throw, true, try, typeof, var, void, while, with, as, implements, interface, let, package, private, protected, public, static, yield, any, boolean, constructor, declare, get, module, require, number, set, string, symbol, type, from, of},
  morecomment=[s]{/*}{*/},
  morecomment=[l]//,
  morestring=[b]",
  morestring=[b]',
  morestring=[b]`
}
\lstdefinelanguage{pegjs}{
  morekeywords={},
  morecomment=[s]{/*}{*/},
  morecomment=[l]//,
  alsolanguage=TypeScript,
  morestring=[b]",
  morestring=[b]',
  morestring=[s]{[}{]},
  alsoletter={/*?=}
  ndkeywords={
    /,
    *,
    ?,
    =
  }
}
\lstdefinelanguage{ReactTypeScript}{
  alsolanguage=TypeScript,
  % HTML
  alsoletter={<>=-},
  ndkeywords={
    <, </, >,
    </a, <a, </a>,
    </abbr, <abbr, </abbr>,
    </address, <address, </address>,
    </area, <area, </area>,
    </area, <area, </area>,
    </article, <article, </article>,
    </aside, <aside, </aside>,
    </audio, <audio, </audio>,
    </audio, <audio, </audio>,
    </b, <b, </b>,
    </base, <base, </base>,
    </bdi, <bdi, </bdi>,
    </bdo, <bdo, </bdo>,
    </blockquote, <blockquote, </blockquote>,
    </body, <body, </body>,
    </br, <br, </br>,
    </button, <button, </button>,
    </canvas, <canvas, </canvas>,
    </caption, <caption, </caption>,
    </cite, <cite, </cite>,
    </code, <code, </code>,
    </col, <col, </col>,
    </colgroup, <colgroup, </colgroup>,
    </data, <data, </data>,
    </datalist, <datalist, </datalist>,
    </dd, <dd, </dd>,
    </del, <del, </del>,
    </details, <details, </details>,
    </dfn, <dfn, </dfn>,
    </div, <div, </div>,
    </dl, <dl, </dl>,
    </dt, <dt, </dt>,
    </em, <em, </em>,
    </embed, <embed, </embed>,
    </fieldset, <fieldset, </fieldset>,
    </figcaption, <figcaption, </figcaption>,
    </figure, <figure, </figure>,
    </footer, <footer, </footer>,
    </form, <form, </form>,
    </h1, <h1, </h1>,
    </h2, <h2, </h2>,
    </h3, <h3, </h3>,
    </h4, <h4, </h4>,
    </h5, <h5, </h5>,
    </h6, <h6, </h6>,
    </head, <head, </head>,
    </header, <header, </header>,
    </hr, <hr, </hr>,
    </html, <html, </html>,
    </i, <i, </i>,
    </iframe, <iframe, </iframe>,
    </img, <img, </img>,
    </input, <input, </input>,
    </ins, <ins, </ins>,
    </kbd, <kbd, </kbd>,
    </keygen, <keygen, </keygen>,
    </label, <label, </label>,
    </legend, <legend, </legend>,
    </li, <li, </li>,
    </link, <link, </link>,
    </main, <main, </main>,
    </map, <map, </map>,
    </mark, <mark, </mark>,
    </math, <math, </math>,
    </menu, <menu, </menu>,
    </menuitem, <menuitem, </menuitem>,
    </meta, <meta, </meta>,
    </meter, <meter, </meter>,
    </nav, <nav, </nav>,
    </noscript, <noscript, </noscript>,
    </object, <object, </object>,
    </ol, <ol, </ol>,
    </optgroup, <optgroup, </optgroup>,
    </option, <option, </option>,
    </output, <output, </output>,
    </p, <p, </p>,
    </param, <param, </param>,
    </pre, <pre, </pre>,
    </progress, <progress, </progress>,
    </q, <q, </q>,
    </rp, <rp, </rp>,
    </rt, <rt, </rt>,
    </ruby, <ruby, </ruby>,
    </s, <s, </s>,
    </samp, <samp, </samp>,
    </script, <script, </script>,
    </section, <section, </section>,
    </select, <select, </select>,
    </small, <small, </small>,
    </source, <source, </source>,
    </span, <span, </span>,
    </strong, <strong, </strong>,
    </style, <style, </style>,
    </summary, <summary, </summary>,
    </sup, <sup, </sup>,
    </svg, <svg, </svg>,
    </table, <table, </table>,
    </tbody, <tbody, </tbody>,
    </td, <td, </td>,
    </template, <template, </template>,
    </textarea, <textarea, </textarea>,
    </tfoot, <tfoot, </tfoot>,
    </th, <th, </th>,
    </thead, <thead, </thead>,
    </time, <time, </time>,
    </title, <title, </title>,
    </tr, <tr, </tr>,
    </track, <track, </track>,
    </u, <u, </u>,
    </ul, <ul, </ul>,
    </var, <var, </var>,
    </video, <video, </video>,
    </wbr, <wbr, </wbr>,
    />, <!
  },
}


\lstset{%
    % Basic design
    backgroundcolor=\color{editorGray},
    basicstyle={\small\ttfamily},
    frame=l,
    % Line numbers
    xleftmargin={0.75cm},
    numbers=left,
    stepnumber=1,
    firstnumber=1,
    numberfirstline=true,
    % Code design
    keywordstyle=\color{blue}\bfseries,
    commentstyle=\color{darkgray}\ttfamily,
    ndkeywordstyle=\color{editorGreen}\bfseries,
    stringstyle=\color{editorOcher},
    % Code
    language=TypeScript,
    alsodigit={.:;},
    tabsize=2,
    showtabs=false,
    showspaces=false,
    showstringspaces=false,
    extendedchars=true,
    breaklines=true,
}

\newcommand{\filelisting}[2][TypeScript]{%
  \subsubsection{\texttt{\protect#2}}
  \lstinputlisting[language=#1, label={lst:#2}]{../#2}
  \pagebreak[3]\-\\
}


\begin{document}

\maketitle

\tableofcontents

\section{The Compiler}

In order to parse SQL commands, we are using a parsing library called \textbf{PEG.js},
which allows us to express a/n SQL syntax as a \emph{Parsing Expression Grammar} (PEG),
and build that grammar into a JavaScript parser.
The grammar was initially structured after Phoenix's SQL grammar,
but generally follows PostgreSQL's syntax and the corresponding ANSI SQL standard.

\subsection{Grammar Rules}

\emph{The grammar is defined in} \verb|src/parser/peg/sql.pegjs|. \\

Parsing starts out with the \verb|Statements| rule, which is a semicolon
delimited list of SQL \verb|Statement|s.
A \verb|Statement| can be either a \verb|Select| or \verb|SelectPair|.
\verb|Select| is broken up into 6 clauses: \verb|TargetClause|, \verb|FromClause|,
\verb|WhereClause|, \verb|GroupByClause|, \verb|HavingClause| and \verb|OrderByClause|.
These correspond to all the possibilities of a valid SQL Select statement.
A \verb|SelectPair| is two seperate \verb|Select| clauses paired together with a
``UNION'', ``INTERSECT'', or ``EXCEPT'' set operation.
You can also apply the ``ALL'' or ``DISTINCT'' modifier to the pair.

The \verb|TargetClause| can have the optional ``DISTINCT'' or ``ALL'' modifier
followed by ``*'' (to allow everything) or a \verb|TargetList|, a
comma-delimited list of \verb|TargetItem|s.
A \verb|TargetItem| is a column-like specifier; it can be a relation name with ``.*''
or an \verb|Operand| with optional alias.

\verb|FromClause| aliases \verb|RelationList|, a list of comma-delimited
relation-like fields, each of which may be a table name (with optional alias)
or a \verb|Join|. A \verb|JOIN| is a pair of relation-like fields joined by
a join-type (``CROSS'', ``INNER'', ``LEFT'', etc) followed by an optional
join-condition (``ON \verb|Condition|'' or ``USING (\verb|TargetList|)'').

\verb|WhereClause| and \verb|HavingClause| are \verb|Condition|s.
The types of \verb|Condition|s are: ``OR'' and ``AND'' (which join two
\verb|Condition|s); comparison, ``LIKE'', and ``BETWEEN'' (which join two
\verb|Operands|); and ``IN'' and ``EXISTS'' (which take \verb|Select|-like
arguments).

\verb|GroupByClause| is simply a \verb|TargetList| like the target clause.
\verb|OrderByClause| is a comma-delimited list of \verb|Operand|s, each optionally
with an ordering-condition (``ASC'', ``DESC'' ``USING \ldots'').

An \verb|Operand| is a \verb|Term| optionally joined to other \verb|Operand|s
by value operations (e.g. arithmetic or concatenation).
A \verb|Term| is a \verb|Literal|, aggregate function, or column reference.
\verb|Literal|s include numeric literals, booleans literals, and string literals
(single-quoted).

A \verb|Name|, which might refer to an operand or relation, is denoted by a
bare-identifier (\verb|/[a-z_][a-z0-9_]*/| and not a \verb|ReservedWord|) or any
string quoted with double-quotes (\verb|"..."|) or backticks (\verb|`...`|).

Both comment forms are supported: starting with \verb|--| and consuming the
rest of the line, and C-style starting with \verb|/*| and ending at \verb|*/|.
Both are permitted anywhere whitespace is.

The \verb|ReservedWord| rule contains 340 keywords that the ISO/ANSI SQL:2008
standard states are \textbf{never allowed as identifiers}. This set is almost
certainly overkill, as most SQL implementations only reserve a \emph{small}
fraction of it. It is also excessively large, making up over $^1/_3$ of the
grammar's sourcecode and $\textbf{90\%}$ of the uncompressed compiled grammar.

\subsection{Interpretation}

\emph{Classes and data structures discussed in this section defined in}
      \verb|src/parser/types.ts|. \\

While parsing the grammar, the PEG.js parser calls JavaScript classes that
correspond to SQL concepts.
These classes include \verb|SqlSelect|, \verb|SqlJoin|, \verb|SqlConditional|,
\verb|SqlLiteral|, etc.
This generates an object-oriented data structure---resembling a tree---that
represents the ``SQL Structure''.

Once the SQL Structure is generated it can be converted into JavaScript classes
that correspond to Relational Algebra concepts.
These classes include \verb|RelRestriction|, \verb|RelProjection|,
\verb|RelJoin|, \verb|RelConditional|, etc.
This generates a data structure---more closely resembling a tree than before---
that represents the ``Relational Algebra Structure''.

\emph{Top-level functions for parsing/conversion defined in} \verb|src/parser/parsing.ts|,
\emph{with conversion implementation functions defined in} \verb|src/parser/sqlToRel.ts|.


\pagebreak

\section{Source Code}

{\large \textbf{All of this code is available at} \verb|https://github.com/SKGrush/sqlparse5300|}

\subsection{\texttt{src/}}

\filelisting[TypeScript]{src/index.ts}
\filelisting[ReactTypeScript]{src/Main.tsx}

\subsection{\texttt{src/components}}

\filelisting[ReactTypeScript]{src/components/QueryInput.tsx}
\filelisting[ReactTypeScript]{src/components/RelationsInput.tsx}
\filelisting[ReactTypeScript]{src/components/TestCase.tsx}
\filelisting[ReactTypeScript]{src/components/Tests.tsx}
\filelisting[ReactTypeScript]{src/components/tree.tsx}

\subsection{\texttt{src/parser}}

\filelisting{src/parser/parsing.ts}
\filelisting[ReactTypeScript]{src/parser/relationalText.tsx}
\filelisting{src/parser/sqlToRel.ts}
\filelisting{src/parser/tests.ts}
\filelisting{src/parser/types.ts}

\subsection{\texttt{src/parser/peg}}

\filelisting[pegjs]{src/parser/peg/sql.pegjs}
\filelisting[pegjs]{src/parser/peg/relations.pegjs}

\subsection{\texttt{src/query\_tree}}

\filelisting[TypeScript]{\detokenize{src/query_tree/node.ts}}
\filelisting[TypeScript]{\detokenize{src/query_tree/operation.tsx}}

\end{document}
